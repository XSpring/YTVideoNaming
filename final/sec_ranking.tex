\subsection{Ranking as Logistic Regression}
We can reformulate the ranking prediction problem between two videos as a binary classification problem. To be specific, let $X_i \in \mathbb{R}^D$ and $X_j \in \mathbb{R}^D$ be feature vectors of two video $i$ and $j$ correspondingly. Each video pair $(i, j)$ is associated with a binary label $Y_{ij}$ defined as follows
\begin{equation}
	Y_{ij} = \begin{cases}
			   1, & \text{if } \text{\#\_of\_views\_i} \geq \text{\#\_of\_views\_j} \\
			   0, & \text{otherwise}
			\end{cases} 
\end{equation}	
 We can form a representative vector of the pair $(i, j)$ as follows
\begin{equation}
	\mathcal{X}_{ij} = k (X_i, X_j),
\end{equation}
where $k: (\mathcal{R}^D, \mathcal{R}^D) \rightarrow \mathcal{R}^{D'}$ is a feature transformation function. There are several options for $k$. 
\begin{itemize}
	\item Difference between two feature vectors: $\mathcal{X}_{ij} = X_i - X_j$
	\item Concatenation of two feature vectors:  $\mathcal{X}_{ij} = [X_i, X_j]$ (Matlab notation)
	\item Kernel functions, e.g. $\mathcal{X}_{ij} = || X_i - X_j ||^2$
\end{itemize}
At the moment, we cannot find any theories/signals to indicate which form of $k$ is the most appropriate. For the scope of the project, we choose to represent $X_{ij}$ as difference between $X_i$ and $X_j$, meaning $D = D'$. The kernelized version is left in future work. 

The function form of classifier $P(Y_{ij}|\mathcal{X}_{ij}, \textbf{w})$ as follows
 \begin{equation}
	 P(Y_{ij}=1|\mathcal{X}_{ij}, \textbf{w}) = \frac{1}{1 + \exp ( w_0 + \sum_d w_d \mathcal{X}_{ij}^d )} = \frac{1}{1 + \exp (\textbf{w}^T \mathcal{X}_{ij})}
 \end{equation}
 
The model parameters $\textbf{w} \in \mathbb{R}^D$ can be learnt using MAP.
\begin{align}
	\label{eqn:objFuncLR}
	\hat{\textbf{w}}_{MAP} &= \operatorname*{arg\,max}_{\textbf{w}} \prod_{(i,j)} P(Y_{ij}| \mathcal{X}_{ij}, \textbf{w}) P(\textbf{w}) \hspace{1 cm} \text{($P(\textbf{w}) \sim \mathcal{N} (0, \tau^2 I) $)}\\ \notag
	&= \operatorname*{arg\,max}_{\textbf{w}} \prod_{\{(i,j)|Y_{ij}=1\}} P(Y_{ij}| \mathcal{X}_{ij}, \textbf{w}) P(\textbf{w}) \hspace{1 cm} \text{($Y_{ij} + Y_{ji} = 1, \mathcal{X}_{ij} = - \mathcal{X}_{ji}$)}\\ \notag
	&= \operatorname*{arg\,max}_{\textbf{w}} \sum_{\{(i,j)|Y_{ij}=1\}} ln P(Y_{ij}| \mathcal{X}_{ij}, \textbf{w}) + ln P(\textbf{w}) \\ \notag
	&= \operatorname*{arg\,max}_{\textbf{w}} \sum_{\{(i,j)|Y_{ij}=1\}} ( (1 - Y_{ij})\textbf{w}^T\mathcal{X}_{ij} - ln(1 + \exp(\textbf{w}^T\mathcal{X}_{ij}))) + \lambda_w \|\textbf{w} \|^2_2 = l(\textbf{w})
\end{align}

We can optimise Equation \ref{eqn:objFuncLR} (a concave function) using Gradient Ascent with the following update rule
\begin{equation}
w^{t+1}_d \leftarrow w^t_d + \eta \frac{\partial l(\textbf{w})}{\partial w_d} = w^t_d + \sum_{\{(i,j)|Y_{ij}=1\}} \mathcal{X}_{ij}^d ( (1 - Y_{ij}) - \frac{\exp(\textbf{w}^T\mathcal{X}_{ij})}{1 + \exp(\textbf{w}^T\mathcal{X}_{ij})})  + \lambda_w w_d
\end{equation}
