Our problem is a special case of the more general \textbf{Learning to rank} problem, which is widely used in many applications such as Information Retrieval, Data Mining, etc. [1] gives a short survey of state-of-the-art approaches to the problem of ranking documents relevant to user queries. Most of them are implemented in Ranklib\footnote{http://people.cs.umass.edu/~vdang/ranklib.html}.

Another form of the problem is ordinal classification (a.k.a. ordinal regression). There are several approaches, mostly adapted from classification techniques, such as support vectors [2], Gaussian processes [3], etc. Ordinal classification is related to the ranking problem in that both involve predicting the objects' orders, and is, like ranking, a supervised learning tasks. The important difference is in the level of orderings. As [1] discussed, learning to rank cares more about the accurate ordering of objects, while the latter focuses on a categorization that happens to be orderable. 
For example, ordinal regression would try to categorize movies into "one star" through "five stars" categories, but there are no specific rankings between movies with the same number of stars, while the learning to rank algorithm has an absolute ordering between all members.
